\documentclass[dvipdfmx, 10pt, aspectratio = 169]{beamer}

\usepackage{graphicx}
\usepackage{amsmath}
\usepackage{bm}
\usepackage{braket}

\usetheme[
	block = fill, 
	progressbar = foot,
	numbering = fraction
]{metropolis}

\setbeamerfont{frame numbering}{size = \normalsize}
\usefonttheme[onlymath]{serif}

\title{Thermodynamics}
\subtitle{1 Thermodynamics Systems}
\date{\today}
\author{Enrico Fermi}

\AtBeginSubsection[]{
   \begin{frame}{目次}
        \tableofcontents[currentsection, 
        subsectionstyle = show/shaded/hide]
    \end{frame}
}

\begin{document}

\begin{frame}
	\titlepage
\end{frame}

\begin{frame}{目次}
\tableofcontents
\end{frame}

\section{1 Thermodaynamic Systems}
\subsection{1.1 The state of a system and its transformation.}

\begin{frame}{title}
\begin{align}
	f(p, V, t) = 0
\end{align}
\end{frame}

\begin{frame}{title}
	\begin{align}
		dL = pSdh
	\end{align}
	
	\begin{align}
		dL = pdV
	\end{align}
	
	\begin{align}
		L = \int^B_A pdV
	\end{align}
	
	\begin{align}
		L = \int^{V_B}_{V_A} pdV
	\end{align}
\end{frame}

\subsection{1.2 Ideal or perfect gases.}

\begin{frame}{title}
	\begin{align}
		pV = \frac{m}{M}RT
	\end{align}
	\begin{align}
		pV = RT
	\end{align}
	\begin{align}
		\rho = \frac{m}{V} = \frac{Mp}{RT}
	\end{align}
\end{frame}

\begin{frame}{title}
	\begin{align*}
		pV = {\rm constant.}
	\end{align*}
	
	\begin{align}
		L = \int^{V_2}_{V_1} pdV &= \frac{m}{M} RT \int^{V_2}_{V_1} \frac{dV}{V} \notag \\
		&= \frac{m}{M}RT \ln \frac{V_2}{V_1} \notag \\
		&= \frac{m}{M} RT \ln \frac{p_1}{p_2}
	\end{align}
	
	\begin{align}
		L = RT \ln \frac{V_2}{V_1} = RT \ln \frac{p_1}{p_2}
	\end{align}
\end{frame}

\begin{frame}{Problems}
\end{frame}

\end{document}
